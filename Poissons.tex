% ----------------------------------------------------------------
% AMS-LaTeX Paper ************************************************
% **** -----------------------------------------------------------
\documentclass[11pt]{amsart}
\usepackage{graphicx, mathabx, amssymb,amsfonts,amsmath,amsthm,newlfont,mathtools}
\usepackage{epsfig,url}
\usepackage[usenames,dvipsnames]{color}
\usepackage{enumerate}
\usepackage[colorlinks=true,linkcolor=red,citecolor=blue]{hyperref}
\usepackage{color}


%Clever ref
\usepackage[noabbrev,capitalize]{cleveref}
\usepackage[all,2cell]{xy} \UseAllTwocells \SilentMatrices

%MARGINS

\setlength{\textwidth}{\paperwidth}
\addtolength{\textwidth}{-2.5in}
\calclayout


% ----------------------------------------------------------------
\vfuzz2pt % Don't report over-full v-boxes if over-edge is small
\hfuzz2pt % Don't report over-full h-boxes if over-edge is small
% THEOREMS -------------------------------------------------------
\newtheorem{thm}{Theorem}[section]
\newtheorem{corollary}[thm]{Corollary}
\newtheorem{lemmma}[thm]{Lemma}
\newtheorem{proposition}[thm]{Proposition}
\newtheorem{Questions}[thm]{Questions}
\theoremstyle{definition}
\newtheorem{definition}[thm]{Definition}

\newtheorem{conjectue}{Conjecture} 
\newtheorem{QQ}{Question} 
\newtheorem{prob}{Problem}
\newtheorem{ex}[thm]{Examples}
\newtheorem{example}[thm]{Example}
\newtheorem{policy}{Policy}
\theoremstyle{remark}
\newtheorem{rem}[thm]{Remark}
\newtheorem{caveat}[thm]{Caveat}
\numberwithin{equation}{section}
% MATH -----------------------------------------------------------
\newcommand{\norm}[1]{\left\Vert#1\right\Vert}
\newcommand{\abs}[1]{\left\vert#1\right\vert}
\newcommand{\set}[1]{\left\{#1\right\}}

\newcommand{\To}{\longrightarrow}
\newcommand*{\Longhookrightarrow}{\ensuremath{\lhook\joinrel\relbar\joinrel\rightarrow}}
\newcommand{\Z}{\mathbb Z}
\newcommand{\Q}{\mathbb Q}
\newcommand{\C}{\mathbb C}
\newcommand{\Ok}{\mathcal O}
\newcommand{\ai}{\mathfrak{a}}
\newcommand{\bi}{\mathfrak{b}}
\newcommand{\R}{\mathbb R}
\newcommand{\N}{\mathbb N}
\newcommand{\AM}{A}
\newcommand{\xx}{\mathsf{x}}
\newcommand{\eqv}{\mathrm{ev}}
\font \rus= wncyr10
\newcommand{\sha}{\, \hbox{\rus x} \,}

\newcommand{\Lie}{\mathrm{Lie}}

\newcommand{\GC}{\mathcal{GC}}
\newcommand{\q}{/\!/}

\newcommand{\tr}{\mathrm{tr}}
\newcommand{\id}{\mathrm{id}}

\newcommand{\can}{\mathrm{can}}

\newcommand{\mm}{\mathfrak{m}}

\newcommand{\GL}{\mathrm{GL}}
\newcommand{\LP}{L}
\newcommand{\FL}{F\!L}
\newcommand{\mc}{\mu}


\newcommand{\0}{\color{blue}{\mathsf{0}}}



\DeclareMathOperator{\Ima}{Im} %Image d'une fonction

%Drapeau européen

\usepackage{graphicx,calc}
\newlength\myheight
\newlength\mydepth
\settototalheight\myheight{Xygp}
\settodepth\mydepth{Xygp}
\setlength\fboxsep{0pt}
\newcommand*\inlinegraphics[1]{%
  \settototalheight\myheight{Xygp}%
  \settodepth\mydepth{Xygp}%
  \raisebox{-\mydepth}{\includegraphics[height=\myheight]{#1}}%
}

%Dessins

\usepackage{tikz}
\usepackage{tikz-cd}
\usepackage{pgfplots}
\usepackage{pgfplotstable}
\tikzset{math3d/.style=
    {x= {(-0.353cm,-0.353cm)}, z={(0cm,1cm)},y={(1cm,0cm)}}}
\tikzset{JLL3d/.style=
    {x= {(0.4cm,-0.2cm)}, z={(0cm,1cm)},y={(-1cm,0cm)}}}
\usetikzlibrary{calc}
\usetikzlibrary{shapes,shapes.geometric,fit,positioning,calc,matrix}
\tikzset{
  optree/.style={scale=.5,thick,grow'=up,level distance=10mm,inner sep=1pt},
  comp/.style={draw=none,circle,fill,line width=0,inner sep=0pt},
  dot/.style={draw,circle,fill,inner sep=0pt,minimum width=3pt},
  circ/.style={draw,circle,inner sep=1pt,minimum width=4mm},
  emptycirc/.style={draw,circle,inner sep=1pt,minimum width=2mm},
  root/.style={level distance=10mm,inner sep=1pt},
  leaf/.style={draw=none,circle,fill,line width=0,inner sep=0pt},
  nodot/.style={draw,circle,inner sep=1pt},
}

\pgfplotsset{compat=1.12}

% ----------------------------------------------------------------

\def\abovespace{\vspace{12pt}}
\def\belowspace{\vspace{8pt}}



\addtolength{\hoffset}{-0.0in} \addtolength{\textwidth}{0in}
\addtolength{\voffset}{-0.0in} \addtolength{\textheight}{0.0in}


% -----------------------------------------------------------------

\title{Poissons}

\author{B\'er\'enice Delcroix-Oger}
\address{}
\email{}

\author{Matthieu Josuat-Verg\`es}
\address{}
\email{}

\author{Guillaume Laplante-Anfossi}
\address{Universit\'e Sorbonne Paris Nord, Laboratoire Analyse, G\'eom\'etrie et Applications, CNRS, UMR 7539, F-93430 Villetaneuse, France.}
\email{laplante-anfossi@math.univ-paris13.fr}



\date{\today}

\subjclass[2010]{Primary [...]; Secondary 18M70} 

\keywords{Polytopes [...]}


\thanks{G. L.-A. was supported by the European Union's Horizon 2020 research and innovation program under the Marie Sklodowska-Curie grant agreement No 754362 \inlinegraphics{EU.png}, by the Natural Sciences and Engineering Research Council of Canada (NSERC) and by the ANR-20-CE40-0016 Higher Algebra, Geometry and Topology.}

\begin{document}


\begin{abstract}
The purpose of this note is to provide a new combinatorial description 
\end{abstract}


\maketitle

\section{Introduction}

TBC

\subsection{Conventions}
We use the notations of \cite{LodayVallette12} for operads.

\section{Poissons}

Let us recall the combinatorial formula for the cellular approximation of the diagonal of the permutahedra from \cite{LA21}.

Let $n\geq 1$, and let us write \[ D(n) \coloneqq \{(I,J) \ | \ I,J\subset\{1,\ldots,n\}, |I|=|J|, I\cap J=\emptyset, \min(I\cup J)\in I \}. \] 

We have
\begin{eqnarray*}
    (\sigma_1,\sigma_2)\in \Ima\triangle_{(P,\vec v)} 
    & \iff & \forall (I,J) \in D(n), \exists l , 
    \left| \left(\bigcup_{1\leq k \leq l} \sigma_1^{k} \right)\cap I \right|
    >
    \left| \left(\bigcup_{1\leq k \leq l} \sigma_1^{k} \right)\cap J \right| \text{ or } \\
    && \exists l' , 
    \left| \left(\bigcup_{1\leq k \leq l'} \sigma_1^{k} \right)\cap I \right|
    <
    \left| \left(\bigcup_{1\leq k \leq l'} \sigma_1^{k} \right)\cap J \right|  \ .
\end{eqnarray*}

We prove the following equivalent description. 

\begin{thm} There exists $(I,J) \in D(n)$ such that $\forall k, |\sigma_1^1\cdots\sigma_1^k \cap I| \leq |\sigma_1^1\cdots\sigma_1^k \cap J|$ and $\forall l, |\sigma_1^1\cdots\sigma_1^l \cap I| \geq |\sigma_1^1\cdots\sigma_1^l \cap J|$ if and only if $\exists (I',J')=(\{i_1,\ldots,i_m\},\{j_1,\ldots,j_m\}) \in D(m)$, $m\leq n$, such that \[\sigma_1 \cap (I'\cup J')=j_1 i_1 j_2 i_2 \cdots j_n i_n \] and \[ \sigma_2 \cap (I'\cup J') = i_2 j_1 i_3 j_2 \cdots i_1 j_n \ , \] where $i_1 = \min (I' \cup J')$. 
\end{thm}

\begin{proof}
    We choose $(I,J)$ minimal such that [...] 
\begin{enumerate}
    \item $\tilde I =\tilde J \coloneqq \emptyset$. 
    We set $i_{*}\coloneqq$ the leftmost $i$ in $\sigma_1 \setminus \tilde I$, $j_{*}\coloneqq$ the leftmost $j$ in $\sigma_2 \setminus \tilde J$.
    Claim: we have $i_{*}=i_1=\min(I)$.
    Proof: If not, $(I\setminus i_{*}, J \setminus j_{*})$ contradict the minimality assumption. 
    \item We set $\tilde I = \{i_1\}, \tilde J = \emptyset$. 
    Claim: we have $\sigma_1 = j_{*}i_1 \cdots$.
    Proof: Idem
    \item We set $\tilde I = \{i_1\}, \tilde J = \{j_1\}$. 
\end{enumerate}
\end{proof}

\emph{Acknowledgements.}    


\bibliographystyle{amsalpha}

\bibliography{Poissons}



\end{document}



