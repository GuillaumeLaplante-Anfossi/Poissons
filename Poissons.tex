% ----------------------------------------------------------------
% AMS-LaTeX Paper ************************************************
% **** -----------------------------------------------------------
\documentclass[11pt]{amsart}
\usepackage{graphicx, mathabx, amssymb,amsfonts,amsmath,amsthm,newlfont,mathtools}
\usepackage{epsfig,url}
\usepackage[usenames,dvipsnames]{color}
\usepackage{enumerate}
\usepackage[colorlinks=true,linkcolor=red,citecolor=blue]{hyperref}
\usepackage{color}
\usepackage{stmaryrd}         % Crochet double barre (entiers)


%Clever ref
\usepackage[noabbrev,capitalize]{cleveref}
\usepackage[all,2cell]{xy} \UseAllTwocells \SilentMatrices

%MARGINS

\setlength{\textwidth}{\paperwidth}
\addtolength{\textwidth}{-2.5in}
\calclayout


% ----------------------------------------------------------------
\vfuzz2pt % Don't report over-full v-boxes if over-edge is small
\hfuzz2pt % Don't report over-full h-boxes if over-edge is small
% THEOREMS -------------------------------------------------------
\newtheorem{thm}{Theorem}[section]
\newtheorem{corollary}[thm]{Corollary}
\newtheorem{lemmma}[thm]{Lemma}
\newtheorem{proposition}[thm]{Proposition}
\newtheorem{Questions}[thm]{Questions}
\theoremstyle{definition}
\newtheorem{definition}[thm]{Definition}

\newtheorem{conjectue}{Conjecture} 
\newtheorem{QQ}{Question} 
\newtheorem{prob}{Problem}
\newtheorem{ex}[thm]{Examples}
\newtheorem{example}[thm]{Example}
\newtheorem{policy}{Policy}
\theoremstyle{remark}
\newtheorem{rem}[thm]{Remark}
\newtheorem{caveat}[thm]{Caveat}
\numberwithin{equation}{section}
% MATH -----------------------------------------------------------
\newcommand{\norm}[1]{\left\Vert#1\right\Vert}
\newcommand{\abs}[1]{\left\vert#1\right\vert}
\newcommand{\set}[1]{\left\{#1\right\}}

\newcommand{\To}{\longrightarrow}
\newcommand*{\Longhookrightarrow}{\ensuremath{\lhook\joinrel\relbar\joinrel\rightarrow}}
\newcommand{\Z}{\mathbb Z}
\newcommand{\Q}{\mathbb Q}
\newcommand{\C}{\mathbb C}
\newcommand{\Ok}{\mathcal O}
\newcommand{\ai}{\mathfrak{a}}
\newcommand{\bi}{\mathfrak{b}}
\newcommand{\R}{\mathbb R}
\newcommand{\N}{\mathbb N}
\newcommand{\AM}{A}
\newcommand{\xx}{\mathsf{x}}
\newcommand{\eqv}{\mathrm{ev}}
\font \rus= wncyr10
\newcommand{\sha}{\, \hbox{\rus x} \,}

\newcommand{\Lie}{\mathrm{Lie}}

\newcommand{\GC}{\mathcal{GC}}
\newcommand{\q}{/\!/}

\newcommand{\tr}{\mathrm{tr}}
\newcommand{\id}{\mathrm{id}}

\newcommand{\can}{\mathrm{can}}

\newcommand{\mm}{\mathfrak{m}}

\newcommand{\GL}{\mathrm{GL}}
\newcommand{\LP}{L}
\newcommand{\FL}{F\!L}
\newcommand{\mc}{\mu}


\newcommand{\0}{\color{blue}{\mathsf{0}}}

%OEIS
\newcommand{\OEIS}[1]{{\rm \href{http://oeis.org/#1}{\texttt{#1}}}}



\DeclareMathOperator{\Ima}{Im} %Image d'une fonction

%Drapeau européen

\usepackage{graphicx,calc}
\newlength\myheight
\newlength\mydepth
\settototalheight\myheight{Xygp}
\settodepth\mydepth{Xygp}
\setlength\fboxsep{0pt}
\newcommand*\inlinegraphics[1]{%
  \settototalheight\myheight{Xygp}%
  \settodepth\mydepth{Xygp}%
  \raisebox{-\mydepth}{\includegraphics[height=\myheight]{#1}}%
}

%Dessins

\usepackage{tikz}
\usepackage{tikz-cd}
\usepackage{pgfplots}
\usepackage{pgfplotstable}
\tikzset{math3d/.style=
    {x= {(-0.353cm,-0.353cm)}, z={(0cm,1cm)},y={(1cm,0cm)}}}
\tikzset{JLL3d/.style=
    {x= {(0.4cm,-0.2cm)}, z={(0cm,1cm)},y={(-1cm,0cm)}}}
\usetikzlibrary{calc}
\usetikzlibrary{shapes,shapes.geometric,fit,positioning,calc,matrix}
\tikzset{
  optree/.style={scale=.5,thick,grow'=up,level distance=10mm,inner sep=1pt},
  comp/.style={draw=none,circle,fill,line width=0,inner sep=0pt},
  dot/.style={draw,circle,fill,inner sep=0pt,minimum width=3pt},
  circ/.style={draw,circle,inner sep=1pt,minimum width=4mm},
  emptycirc/.style={draw,circle,inner sep=1pt,minimum width=2mm},
  root/.style={level distance=10mm,inner sep=1pt},
  leaf/.style={draw=none,circle,fill,line width=0,inner sep=0pt},
  nodot/.style={draw,circle,inner sep=1pt},
}

\pgfplotsset{compat=1.12}

% ----------------------------------------------------------------

\def\abovespace{\vspace{12pt}}
\def\belowspace{\vspace{8pt}}



\addtolength{\hoffset}{-0.0in} \addtolength{\textwidth}{0in}
\addtolength{\voffset}{-0.0in} \addtolength{\textheight}{0.0in}


% -----------------------------------------------------------------

\title{Poissons}

\author{B\'er\'enice Delcroix-Oger}
\address{}
\email{}

\author{Matthieu Josuat-Verg\`es}
\address{}
\email{}

\author{Guillaume Laplante-Anfossi}
\address{School of Mathematics and Statistics, University of Melbourne, Victoria, Australia}
\email{guillaume.laplanteanfossi@unimelb.edu.au}

\author{Kurt Stoeckl}
\address{}
\email{}



\date{\today}

\subjclass[2010]{Primary [...]; Secondary 18M70} 

\keywords{Polytopes [...]}


\thanks{G. L.-A. was supported by the European Union's Horizon 2020 research and innovation program under the Marie Sklodowska-Curie grant agreement No 754362 \inlinegraphics{EU.png}, by the Natural Sciences and Engineering Research Council of Canada (NSERC) and by the ANR-20-CE40-0016 Higher Algebra, Geometry and Topology.}

\begin{document}


\begin{abstract}
The purpose of this note is to provide a new combinatorial description 
\end{abstract}


\maketitle

\section{Introduction}

TBC

%\subsection{Conventions}
%We use the notations of \cite{LodayVallette12} for operads.

\section{Poissons}

Let us recall the combinatorial formula for the cellular approximation of the diagonal of the permutahedra from \cite{LA21}.

Let $n\geq 1$, and let us write \[ D(n) \coloneqq \{(I,J) \ | \ I,J\subset\{1,\ldots,n\}, |I|=|J|, I\cap J=\emptyset, \min(I\cup J)\in I \}. \] 

We have
\begin{eqnarray*}
    (\sigma_1,\sigma_2)\in \Ima\triangle_{(P,\vec v)} 
    & \iff & \forall (I,J) \in D(n), \exists l , 
    \left| \left(\bigcup_{1\leq k \leq l} \sigma_1^{k} \right)\cap I \right|
    >
    \left| \left(\bigcup_{1\leq k \leq l} \sigma_1^{k} \right)\cap J \right| \text{ or } \\
    && \exists l' , 
    \left| \left(\bigcup_{1\leq k \leq l'} \sigma_1^{k} \right)\cap I \right|
    <
    \left| \left(\bigcup_{1\leq k \leq l'} \sigma_1^{k} \right)\cap J \right|  \ .
\end{eqnarray*}

We prove the following equivalent description. 

\begin{thm} There exists $(I,J) \in D(n)$ such that $\forall k, |\sigma_1^1\cdots\sigma_1^k \cap I| \leq |\sigma_1^1\cdots\sigma_1^k \cap J|$ and $\forall l, |\sigma_1^1\cdots\sigma_1^l \cap I| \geq |\sigma_1^1\cdots\sigma_1^l \cap J|$ (diagonal condition) if and only if $\exists (I',J')=(\{i_1,\ldots,i_m\},\{j_1,\ldots,j_m\}) \in D(m)$, $m\leq n$, such that \[\sigma_1 \cap (I'\cup J')=j_1 i_1 j_2 i_2 \cdots j_n i_n \] and \[ \sigma_2 \cap (I'\cup J') = i_2 j_1 i_3 j_2 \cdots i_n j_{n-1} i_1 j_n \ , \] where $i_1 = \min (I' \cup J')$ (fish condition). 
\end{thm}

\begin{proof}
\begin{itemize}
\item If a pair of permutations $(\sigma_1, \sigma_2) \in   \mathfrak{S}_N^2$ satisfies the fish condition, then there exist two sets $I$ and $J$ of same cardinality such that $\min(I)<\min(J)$. Denoting $\sigma_1$ and $\sigma_2$ by two words of size $N$ $\sigma_1^1 \ldots \sigma_1^N$ and $\sigma_2^1 \ldots \sigma_2^N$, then the pair $((\sigma_1, \sigma_2), (I,J))$ satisfies that for any $k$ in $\llbracket 1;N\rrbracket$, $|\sigma_1^1 \ldots \sigma_1^k \cap J| \geq |\sigma_1^1 \ldots \sigma_1^k \cap I|$ and $|\sigma_2^1 \ldots \sigma_2^k \cap I| \geq |\sigma_2^1 \ldots \sigma_2^k \cap J|$, hence the diagonal condition.
\item We will now prove the converse. Let us presume that $(\sigma_1, \sigma_2)$ is a pair of permutations satisfying the diagonal condition for a pair of sets $(I,J) \in D(n)$, minimal for the inclusion of sets.
\begin{description}
\item[Case $n=1$] 
\end{description}
If $|I|=|J|=1$, then it follows directly from the diagonal condition above that ${\sigma_1}_{| I \cup J}=j_1 i_1$ and ${\sigma_1}_{|I \cup J}=i_1 j_1$, hence the fish condition is satisfied.
\begin{description}
\item[Case $n>1$] 
\end{description}
In this case, the proof is made by absurdum 
by considering the number of "well-placed" elements of $I$ and $J$ in $\sigma_1$ and $\sigma_2$. In what follows, for any set $E$, $\sigma^{E}_i$ will stands for $(\sigma_i)_{|E}$. We write also $n_{i,k}^E$ for the number of elements of $E$ in the $k$ first letters of $\sigma_i$. The main argument in each of the small proofs below is the same: if the permutations do not satisfy the pattern described above, then it is possible to find an appropriate pair of elements $(i,j)\in I \times J$ such that $(I-i,J-j)$ satisfies the diagonal condition, hence  contradicting the minimality of $(I,J)$.

We first prove that the leftmost element of $\sigma^{I}_1$ is $i_1$. Indeed, if it is not the case, we consider $i$, the leftmost element in $\sigma^{I}_1$ and $j$ the leftmost element in $\sigma^{J}_2$. The pair $(I-i,J-j)$ is in $D(n-1)$, as $i$ is different from $i_1$. Moreover, it is clear that the diagonal condition still holds for $((\sigma_1, \sigma_2), (I,J))$. As this would contradict the minimality of $(I,J)$, the leftmost element of $\sigma^{I}_1$ is $i_1$.

We then prove that $\sigma^{I \cup J}_1$ starts by $j_1 i_1$ and that this $j_1$ is exactly the leftmost element in  $\sigma^{J}_2$. On that purpose, we suppose that either  $i_1$ is preceeded by several elements of $J$ or that the unique element of $J$ is not the leftmost one in $\sigma^{J}_2$. We then adapt the previous argument by choosing $i$ to be the leftmost element in $\sigma^{I-\{i_1\}}_1$ and $j$ the leftmost element in $\sigma^{J}_2$. The pair $(I-i,J-j)$ is in $D(n-1)$. Let us briefly explain while  the diagonal condition would still be fulfilled in this case. If $j$ is after $i_1$ in $\sigma_1$, then the difference $n_{1,k}^{J-j}-n_{1,k}^{I-i}$ is greater than $n_{1,k}^{J}-n_{1,k}^{I}$ for any $k$, hence is non negative. If $j$ is before $i_1$ in $\sigma_1$, then by hypothesis, the difference $n_{1,k}^{J-j}-n_{1,k}^{I-i}$ is:
\begin{itemize}
\item strictly positive before $i_1$ an greater than $1$ just before $i_1$
\item non negative after $i_1$
\item increase between $i_1$ and $i$
\item is equal to $n_{1,k}^{J}-n_{1,k}^{I}$ after $i$,
\end{itemize} 
hence is always non negative.
Moreover, if $i$ is after $j$ in $\sigma_2$, the diagonal condition is clearly satisfied. If $i$ is before $j$, then the difference $n_{2,k}^{I-i}-n_{1,k}^{J-j}$ is:
\begin{itemize}
\item strictly positive before $j$ an greater than $1$ just before $j$
\item is equal to $n_{2,k}^{I}-n_{1,k}^{J}$ after $j$,
\end{itemize} 
hence is always non negative. In short, if $i_1$ is preceeded by several elements of $J$ or the unique element of $J$ is not the leftmost one in $\sigma^{J}_2$, we obtain a contradiction with the minimality of $(I,J)$.

Let us now consider the biggest $k\geq 1$ such that $\sigma^{I \cup J}_1$ begins with $j_1 i_1 j_2 i_2 \ldots j_k i_k$ and $\sigma^{I \cup J}_2$ begins with $i_2 j_1 i_3 j_2\ldots i_k j_{k-1} w j_k$, where $w$ is a word with letters in $I$. We want to show that $k=n$. Let us first remark that if $k=n$, $w=i_1$. If $1\leq k<n$, then the sets $\tilde{I}=I-\{i_1, \ldots, i_k\}$ and $\tilde{J}=J-\{j_1, \ldots, j_k\}$ are non empty. Let us choose $i_{k+1}$ to be the leftmost element in $\sigma^{\tilde{I}}_1$ and $j_{k+1}$ the leftmost element in $\sigma^{\tilde{J}}_2$. We thus have $\sigma^{I \cup J}_1=j_1 i_1 j_2 i_2 \ldots j_k i_k w' i_{k+1}\ldots$, where $w'$ is in $J$ and $\sigma^{I \cup J}_2= i_2 j_1 i_3 j_2\ldots i_k j_{k-1} w j_k w'' j_{k+1}\ldots $, where $w$ and $w'$ are words with letters in $I$. The pair $(I-i_{k+1},J-j_{k+1})$ is in $D(n-1)$. Following the study as in the previous case, $\sigma_1$ always satisfies the diagonal condition for $(I-i_{k+1},J-j_{k+1})$ and $\sigma_2$ satisfies it if and only if $w \neq i$. By minimality of $(I,J)$, we then have $w=i_{k+1}$. If $k+1=n$, we are done as the only possible word in $J$ is $j_{k+1}$, hence $w'=j_{k+1}$. Otherwise, we can choose $i_{k+2}$ to be the leftmost element in $\sigma_1^{\tilde{I}-i_{k+1}}$. Using the same reasoning as above, we show that $((\sigma_1, \sigma_2),(I-i_{k+2},J-j_{k+1}))$ satisfies the diagonal condition if and only if $w'\neq j_{k+1}$. To sum up, the only possibility for $(I,J)$ to be minimal is to have $k=n$, which implies the fish condition.
\end{itemize}
\end{proof}

\begin{corollary} For any pair of permutations $(\sigma_1, \sigma_2$, there exists $(I,J) \in D(n)$ such that $((\sigma_1, \sigma_2),(I,J))$ satisfies the diagonal condition if and only if there exists $(I',J') \in E(m)$, $m<n$ such that $((\sigma_1, \sigma_2),(I',J'))$ satisfies the fish condition, with 
\begin{multline}
E(m)=\{(I,J)\in D(m)| \min(J)<\min(I-\min(I)), |\llbracket 1; k \rrbracket \cap J| > |\llbracket 1; k \rrbracket \cap I| \\ \text{ if } |\llbracket 1; k \rrbracket \cap J| \geq 2 \text{ and } I \subsetneq \llbracket 1; k \rrbracket \}
\end{multline}
\end{corollary}

\begin{proof}
It follows directly from the fish condition: if the fish condition is satisfied, as inversions of $\sigma_1$ are included in inversions of $\sigma_2$, we get $j_{k-1},j_k<i_k$ for any $k>1$.
\end{proof}

\begin{figure}[h]
\centerline{
\begin{tabular}{r|c}
\textbf{dim} & \textbf{0}  \\
\hline
\textbf{0} & 1  
\end{tabular}
\ \ \
\begin{tabular}{r|cc}
\textbf{dim} & \textbf{0} & \textbf{1}  \\
\hline
\textbf{0} & 3 & 1 \\
\textbf{1} & 1 &  
\end{tabular}
\ \ \
\begin{tabular}{r|ccc}
\textbf{dim} & \textbf{0} & \textbf{1} & \textbf{2}  \\
\hline
\textbf{0} & 17 & 12 & 1 \\
\textbf{1} & 12 &  6 & \\
\textbf{2} & 1 &  & 
\end{tabular}
\ \ \
\begin{tabular}{r|cccc}
\textbf{dim} & \textbf{0} & \textbf{1} & \textbf{2} & \textbf{3} \\
\hline
\textbf{0} & 149 & 162 & 38 & 1 \\
\textbf{1} & 162 & 150 & 24 & \\
\textbf{2} & 38 & 24 & & \\
\textbf{3} & 1 & & &
\end{tabular}
}
\caption{Number of pairs of faces in the cellular image of the diagonal $0$, $1$, $2$ and $3$-dimensional permutahedra.}
\label{t:dim1-3}
\end{figure}

\begin{figure}[h]
\centerline{
\begin{tabular}{r|ccccc}
\textbf{dim} & \textbf{0} & \textbf{1} & \textbf{2} & \textbf{3} & \textbf{4} \\
\hline
\textbf{0} & 1809 & 2660 & 1080 & 110 & 1 \\
\textbf{1} & 2660 & 3540 & 1200 & 80 & \\
\textbf{2} & 1080 & 1200 & 270 & & \\
\textbf{3} & 110 & 80 & && \\
\textbf{4} & 1 & & & &
\end{tabular}
\ \ \
\begin{tabular}{r|cccccc}
\textbf{dim} & \textbf{0} & \textbf{1} & \textbf{2} & \textbf{3} & \textbf{4} & \textbf{5} \\
\hline
\textbf{0} & 28399 & 52635 & 30820 & 6165 & 302 & 1 \\
\textbf{1} & 52635 & 90870 & 67580 & 7785 & 240 & \\
\textbf{2} & 30820 & 47580 & 20480 & 2160 & & \\
\textbf{3} & 6165 & 7785 & 2160 & && \\
\textbf{4} & 302 & 240 & & &&\\
\textbf{5} & 1 & & & &&
\end{tabular}
}
\caption{Number of pairs of faces in the cellular image of the diagonal $4$ and $5$-dimensional permutahedra.}
\label{t:dim4-5}
\end{figure}

The diagonals obey to the formula $\binom{n}{k+1}(k+1)^{n-k-1}(n-k)^{k-1}$.

Their sum is equal to $1,2,8,50,432,4802,65536...$.

\begin{figure}[h]
\centerline{\begin{tabular}{c|c|rrrrrrr|l}
\textbf{Pairs $(F,G) \in \Ima\triangle_{(P,\vec v)}$} & \textbf{Polytopes} & \textbf{0} & \textbf{1} & \textbf{2} & \textbf{3} & \textbf{4} & \textbf{5} & \textbf{6} & \textbf{\cite{OEIS}} \\
\hline
& \text{Associahedra} & 1 & 2 & 6 & 22 & 91 & 408 & 1938 & \OEIS{A000139}  \\
$\dim F + \dim G = \dim P$  & \text{Multiplihedra} & 1 & 2 & 8 & 42 & 254 & 1678 & 11790 &  to appear \\
  & \text{Permutahedra} & 1 & 2 & 8 & 50 & 432 & 4802 & 65536 &  \OEIS{A007334} \\
\hline
  & \text{Associahedra} & 1 & 3 & 13 & 68 & 399 & 2530 & 16965 &  \OEIS{A000260} \\
  $\dim F=\dim G =0$ & \text{Multiplihedra} & 1 & 3 & 17 & 122 & 992 & 8721 & 80920 & to appear \\
  & \text{Permutahedra} & 1 & 3 & 17 & 149 & 1809 & 28399 & 550297 &  \OEIS{A213507} 
\end{tabular}}
\caption{Number of pairs of faces in the cellular image of the diagonal of the associahedra, multiplihedra and permutahedra of dimension $0\leq \dim P \leq 6$, induced by any good orientation vector.}
\label{table:numerology}
\end{figure}



\emph{Acknowledgements.}    


\bibliographystyle{amsalpha}

\bibliography{Poissons}



\end{document}



