% !TEX root = ../Poissons.tex




%Clever ref
\usepackage[noabbrev,capitalize]{cleveref}
\usepackage[all,2cell]{xy} \UseAllTwocells \SilentMatrices

%% For algorithms %% following tutorial from Shanto Roy
\usepackage[linesnumbered,ruled,vlined]{algorithm2e}
        %%% Coloring the comment as blue
\newcommand\mycommfont[1]{\footnotesize\ttfamily\textcolor{blue}{#1}}
\SetCommentSty{mycommfont}
\SetKwInput{KwInput}{Input}                % Set the Input
\SetKwInput{KwOutput}{Output}              % set the Output


%MARGINS

\setlength{\textwidth}{\paperwidth}
\addtolength{\textwidth}{-2.5in}
\calclayout


% ----------------------------------------------------------------
\vfuzz2pt % Don't report over-full v-boxes if over-edge is small
\hfuzz2pt % Don't report over-full h-boxes if over-edge is small
% THEOREMS -------------------------------------------------------
\newtheorem{thm}{Theorem}[section]
\newtheorem{corollary}[thm]{Corollary}
\newtheorem{lemma}[thm]{Lemma}
\newtheorem{construction}[thm]{Construction}
\newtheorem{proposition}[thm]{Proposition}
\newtheorem{Questions}[thm]{Questions}
\theoremstyle{definition}
\newtheorem{definition}[thm]{Definition}

\newtheorem{conjectue}{Conjecture} 
\newtheorem{QQ}{Question} 
\newtheorem{prob}{Problem}
\newtheorem{ex}[thm]{Examples}
\newtheorem{example}[thm]{Example}
\newtheorem{policy}{Policy}
\theoremstyle{remark}
\newtheorem{rem}[thm]{Remark}
\newtheorem{caveat}[thm]{Caveat}
\numberwithin{equation}{section}
% MATH -----------------------------------------------------------
\newcommand{\norm}[1]{\left\Vert#1\right\Vert}
\newcommand{\abs}[1]{\left\vert#1\right\vert}
\newcommand{\set}[1]{\left\{#1\right\}}

\newcommand{\To}{\longrightarrow}
\newcommand*{\Longhookrightarrow}{\ensuremath{\lhook\joinrel\relbar\joinrel\rightarrow}}
\newcommand{\Z}{\mathbb Z}
\newcommand{\Q}{\mathbb Q}
\newcommand{\C}{\mathbb C}
\newcommand{\Ok}{\mathcal O}
\newcommand{\ai}{\mathfrak{a}}
\newcommand{\bi}{\mathfrak{b}}
\newcommand{\R}{\mathbb R}
\newcommand{\N}{\mathbb N}
\newcommand{\AM}{A}
\newcommand{\xx}{\mathsf{x}}
\newcommand{\eqv}{\mathrm{ev}}
\font \rus= wncyr10
\newcommand{\sha}{\, \hbox{\rus x} \,}

\newcommand{\Lie}{\mathrm{Lie}}

\newcommand{\GC}{\mathcal{GC}}
\newcommand{\q}{/\!/}

\newcommand{\tr}{\mathrm{tr}}
\newcommand{\id}{\mathrm{id}}

\newcommand{\can}{\mathrm{can}}

\newcommand{\mm}{\mathfrak{m}}

\newcommand{\GL}{\mathrm{GL}}
\newcommand{\LP}{L}
\newcommand{\FL}{F\!L}
\newcommand{\mc}{\mu}

%Important sets 
\newcommand{\EC}{\mathcal{E}} %essential complimentary partitions
\newcommand{\OP}{\triangle} %ordered partitions
\newcommand{\BT}{\mathcal{B}} %bipartite trees


\newcommand{\0}{\color{blue}{\mathsf{0}}}

%OEIS
\newcommand{\OEIS}[1]{{\rm \href{http://oeis.org/#1}{\texttt{#1}}}}

%Commentaires 

\newcommand{\Guillaume}[1]{\textcolor{magenta}{\underline{Guillaume}: #1}}
\newcommand{\Kurt}[1]{\textcolor{blue}{\underline{Kurt}: #1}}
\newcommand{\BDO}[1]{\textcolor{ForestGreen}{\underline{B\'er\'enice}: #1}}

\DeclareMathOperator{\Ima}{Im} %Image d'une fonction

%Drapeau européen

\usepackage{graphicx,calc}
\newlength\myheight
\newlength\mydepth
\settototalheight\myheight{Xygp}
\settodepth\mydepth{Xygp}
\setlength\fboxsep{0pt}
\newcommand*\inlinegraphics[1]{%
  \settototalheight\myheight{Xygp}%
  \settodepth\mydepth{Xygp}%
  \raisebox{-\mydepth}{\includegraphics[height=\myheight]{#1}}%
}

%Dessins

\usepackage{tikz}
\usepackage{tikz-cd}
\usepackage{pgfplots}
\usepackage{pgfplotstable}
\tikzset{math3d/.style=
    {x= {(-0.353cm,-0.353cm)}, z={(0cm,1cm)},y={(1cm,0cm)}}}
\tikzset{JLL3d/.style=
    {x= {(0.4cm,-0.2cm)}, z={(0cm,1cm)},y={(-1cm,0cm)}}}
\usetikzlibrary{calc}
\usetikzlibrary{shapes,shapes.geometric,fit,positioning,calc,matrix}
\tikzset{
  optree/.style={scale=.5,thick,grow'=up,level distance=10mm,inner sep=1pt},
  comp/.style={draw=none,circle,fill,line width=0,inner sep=0pt},
  dot/.style={draw,circle,fill,inner sep=0pt,minimum width=3pt},
  circ/.style={draw,circle,inner sep=1pt,minimum width=4mm},
  emptycirc/.style={draw,circle,inner sep=1pt,minimum width=2mm},
  root/.style={level distance=10mm,inner sep=1pt},
  leaf/.style={draw=none,circle,fill,line width=0,inner sep=0pt},
  nodot/.style={draw,circle,inner sep=1pt},
}

\pgfplotsset{compat=1.12}

% ----------------------------------------------------------------

\def\abovespace{\vspace{12pt}}
\def\belowspace{\vspace{8pt}}



\addtolength{\hoffset}{-0.0in} \addtolength{\textwidth}{0in}
\addtolength{\voffset}{-0.0in} \addtolength{\textheight}{0.0in}

