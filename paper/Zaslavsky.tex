 

\documentclass[11pt,leqno]{amsart} 
\usepackage{etex}
\usepackage[all]{xy}
%\usepackage{mathspec}

\newcommand{\mathbbm}[1]{\text{\usefont{U}{bbm}{m}{n}#1}}

%%%% General purpose %%%%
\usepackage[latin1]{inputenc}         % Gestion of l'utf8
%\usepackage{lmodern}                % Fontes vectoris?es
%\usepackage[T1]{fontenc}            % Accentuation and fontes r?centes
\usepackage[french,english]{babel}  % Langues of l'environnement
\usepackage{ucs}                      % Caract?res unicode
%\usepackage{showkeys}              % Affiche les cl?s of citation
%\listfiles                         % Affiche dans le fichier log les versions des paquets
%\overfullrule=5pt                   % Affiche un carr? noir l?  o? il y a des badbox

%%%% Affichage des figures, graphiques and autres %%%%
\usepackage{graphicx}         % Am?lioration du paquet graphics
\usepackage{xcolor}           % Am?lioration du paquet color
\usepackage{enumitem}         % Gestion personnalis? des label d'enumerate
\usepackage{tikz}
%\usetikzlibrary{fit,calc,positioning,decorations.pathreplacing,matrix, shapes}
%\usetikzlibrary{trees}


%%%% Paquets math?matiques %%%%
\usepackage{amsmath, amsthm}  % Commandes of l'ams
\usepackage{amsfonts, amssymb}
\usepackage{stmaryrd}         % Crochet double barre (entiers)
\usepackage{bm}               % Lettres grecques en gras

%%%% Configuration of hyperref %%%%
\usepackage[ breaklinks,  % Permet les liens vers les titres longs
            pagebackref,           % Ancres des r?f?rences vers les pages les citant
            pdfborder={0 0 0}
            ]{hyperref}  % Liens hypertextes vers r?f?rences


\usepackage[nohints]{minitoc} % Possibilit? of cr?er des mini tables des mati?res par chapitre

\usepackage{xargs}

\newcommand{\OEIS}[1]{\cite[{\rm \href{http://oeis.org/#1}{\texttt{#1}}}]{OEIS}}

%%%% Environnement math?matique %%%%
\theoremstyle{definition}
\newtheorem{definition}{Definition}[subsection]
\newtheorem{remark}[definition]{Remark}
\newtheorem{example}[definition]{Example}
\newtheorem{examples}[definition]{Examples}
\newtheorem{acknowledgement}[definition]{Acknowledgement}
\newtheorem{notation}[definition]{Notation}
\theoremstyle{plain}
\newtheorem{lemma}[definition]{Lemma}
\newtheorem{proposition}[definition]{Proposition}
\newtheorem{corollary}[definition]{Corollary}
\newtheorem{theorem}[definition]{Theorem}


% math special letters
\newcommand{\R}{\mathbb{R}} % reals
\newcommand{\N}{\mathbb{N}} % naturals
\newcommand{\Z}{\mathbb{Z}} % integers
\newcommand{\C}{\mathbb{C}} % complex
\newcommand{\I}{\mathbb{I}} % set of integers
\newcommand{\K}{\mathbb{K}} % field
\newcommand{\fA}{\mathfrak{A}} % alternating group
\newcommand{\fB}{\mathfrak{S}^\textsc{b}} % signed symmetric group
\newcommand{\cA}{\mathcal{A}} % algebra
\newcommand{\cC}{\mathcal{C}} % collection
\newcommand{\cS}{\mathcal{S}} % ground set
\newcommand{\uR}{\underline{R}} % underline set
\newcommand{\uS}{\underline{S}} % underline set
\newcommand{\uT}{\underline{T}} % underline set
\newcommand{\oS}{\overline{S}} % overline set
\newcommand{\ucS}{\underline{\cS}} % underline ground set
\renewcommand{\b}[1]{{\boldsymbol{#1}}} % bold letters
\newcommand{\bb}[1]{\mathbb{#1}} % bb letters
\newcommand{\f}[1]{\mathfrak{#1}} % frak letters
\newcommand{\h}{\widehat} % hat letters

%%%% Quelques operateurs utiles %%%%
\DeclareMathOperator{\vect}{Span}
\DeclareMathOperator{\prim}{Prim}

%billboard
\renewcommand{\b}[1]{\boldsymbol{#1}} % bold
\newcommandx{\arrangement}[1][1 = A]{\mathcal{#1}} % arrangement
\newcommand{\HH}{\mathbb{H}} % hyperplane
\newcommandx{\braidArrangement}[1][1 = n]{\arrangement[B]_{#1}} % braid arrangement
\newcommandx{\multiBraidArrangement}[2][1 = n, 2 = \ell]{\arrangement[B]_{#1}^{#2}} % (l,n)-braid arrangement
\newcommandx{\rank}[1][1 = \arrangement]{\operatorname{rk}(#1)} % rank
\newcommandx{\facePoset}[1][1 = \arrangement]{\mathsf{Fa}(#1)} % face poset
\newcommandx{\fPol}[2][1 = \arrangement, 2 = x]{\b{f}_{#1}(#2)} % face polynomial
\newcommandx{\bPol}[2][1 = \arrangement, 2 = x]{\b{b}_{#1}(#2)} % bounded face polynomial
\newcommandx{\flatPoset}[1][1 = \arrangement]{\mathsf{Fl}(#1)} % intersection poset
\newcommandx{\charPol}[2][1 = \arrangement, 2 = y]{\b{\chi}_{#1}(#2)} % characteristic polynomial
\newcommandx{\mobPol}[3][1 = \arrangement, 2 = x, 3 = y]{\b{\mu}_{#1}(#2, #3)} % Mobius polynomial
\newcommandx{\partitionPoset}[1][1 = n]{\mathsf{Pa}_{#1}} % partition poset
\newcommandx{\forestPoset}[2][1 = n, 2 = \ell]{\mathsf{Fo}_{#1}^{#2}} % forest poset
\newcommandx{\rainbowForests}[2][1 = n, 2 = \ell]{\mathsf{RF}_{#1}^{#2}} % rainbow forests
\newcommandx{\rainbowTrees}[2][1 = n, 2 = \ell]{\mathsf{RT}_{#1}^{#2}} % rainbow trees
\newcommandx{\Perm}[1][1 = n]{\mathds{P}\mathsf{erm}(#1)} % permutahedron

% math commands
\newcommand{\set}[2]{\left\{ #1 \;\middle|\; #2 \right\}} % set notation
\newcommand{\bigset}[2]{\big\{ #1 \;\big|\; #2 \big\}} % big set notation
\newcommand{\Bigset}[2]{\Big\{ #1 \;\Big|\; #2 \Big\}} % Big set notation
\newcommand{\setangle}[2]{\left\langle #1 \;\middle|\; #2 \right\rangle} % set notation
\newcommand{\ssm}{\smallsetminus} % small set minus
\newcommand{\dotprod}[2]{\left\langle \, #1 \; \middle| \; #2 \, \right\rangle} % dot product
\newcommand{\symdif}{\,\triangle\,} % symmetric difference
\newcommand{\one}{\b{1}} % the all one vector
\newcommand{\eqdef}{\mbox{\,\raisebox{0.2ex}{\scriptsize\ensuremath{\mathrm:}}\ensuremath{=}\,}} % :=
\newcommand{\defeq}{\mbox{~\ensuremath{=}\raisebox{0.2ex}{\scriptsize\ensuremath{\mathrm:}} }} % =:
\newcommand{\simplex}{\b{\triangle}} % simplex
\renewcommand{\implies}{\Rightarrow} % imply sign
\newcommand{\transpose}[1]{{#1}^t} % transpose matrix
\renewcommand{\complement}[1]{\bar{#1}} % complement

\usepackage[color=blue!50]{todonotes}

\usepackage{xargs}


\usepackage{endnotes}
\usepackage{stmaryrd}
\usepackage{textcomp}


\title{Application of Zaslavsky theorem}


\author{}

 \date{\today}


\begin{document}
\maketitle

Let us recall that the M\"obius polynomial of the braid arrangement~$\braidArrangement$ is given by
\[
\mobPol[\braidArrangement] = \sum_{k \in [n]} x^{k-1} S(n,k) \prod_{i \in [k-1]} (y-i) \, ,
\].

We denote by $P_n(x)$ the polynomial $\mobPol[\braidArrangement][x][0]$. The polynomial $P_n(x)$ rewrites as:
\begin{equation*}
P_n(x)=\sum_{k=1}^n (-1)^{k-1} (k-1)! S(n,k) x^{k-1}. 
\end{equation*}

The coefficients of this polynomial is given by the sequence \OEIS{A028246}.

\begin{lemma}
This polynomial is a product of $(1-x)$ with the f-polynomial of the permutohedron, whose coefficients are given by the sequence \OEIS{A019538}. 

\end{lemma}

\begin{proof}
This lemma is equivalent to the following equality:
\begin{equation*}
\sum_{k=1}^n (-1)^{k-1} (k-1)! S(n,k) x^{k-1} = (1-x) \times \sum_{k=1}^{n-1} (-1)^{k-1} k! S(n-1,k) x^{k-1}
\end{equation*}
Distributing $(1-x)$ in the RHS gives:
\begin{multline*}
(1-x) \times \sum_{k=1}^{n-1} (-1)^{k-1}  k! S(n-1,k) x^{k-1} = \sum_{k=1}^{n-1} k! S(n-1,k) (-x)^{k-1} + \sum_{k=1}^{n-1} k! S(n-1,k) (-x)^{k} \\
= \sum_{k=1}^{n-1} k! S(n-1,k) (-x)^{k-1} + \sum_{k=2}^{n} (k-1)! S(n-1,k-1) (-x)^{k-1} + (n-1)! S(n-1, n-1) (-x)^{n-1} \\
= S(n-1,1) (-x)^0 + \sum_{k=2}^{n-1} (k-1)! \left( S(n-1,k-1) + k S(n-1,k) \right) (-x)^{k-1}
\end{multline*}
The following inductive formula on Stirling numbers of the second kind holds for $0<k<n$ :
\begin{equation}
S(n+1,k) = k S(n,k) + S(n,k-1)
\end{equation}
Hence the result.
\end{proof}

We now compute the M\"obius polynomial of the $(\ell,n)$-braid arrangement~$\multiBraidArrangement$:

\begin{equation*}
\mobPol[\multiBraidArrangement] =\sum_{F \leq G} \mu_{\flatPoset}(F,G) \, x^{\dim(F)} \, y^{\dim(G)},
\end{equation*}
where the sum runs over all intervals of the $(\ell,n)$-rainbow forest poset~$\forestPoset$.

We have:

\begin{align*}
\mobPol[\multiBraidArrangement] & =\sum_{F \leq G} \mu_{\flatPoset}(F,G) \, x^{\dim(F)} \, y^{\dim(G)}, \\
 &=\sum_{G \in \rainbowForests} y^{n-1 - |E(G)| - \ell} \sum_{F \in \rainbowForests, F \leq G } \mu_{\forestPoset}(F,G) \, x^{n-1 - |E(F)| - \ell}, \\
 &=\sum_{G \in \forestPoset} y^{n-\ell n - 1 + \sum_{i=1}^\ell |G_i|} \sum_{F \in \forestPoset, F \leq G } \mu_{\forestPoset}(F,G) \, x^{n-\ell n - 1 + \sum_{i=1}^\ell |F_i|}, \\ 
\end{align*}
where $E(G)$ (resp. $E(G,i)$) denotes the set of edges (resp. edges of color $i$) in $G$ which satisfies $n-|E(G,i)| =  |G_i|$.

Observe that for~$\b{F} \eqdef (F_1, \dots, F_\ell)$ and~$\b{G} \eqdef (G_1, \dots, G_\ell)$ in~$\forestPoset$, we have
\[
[\b{F}, \b{G}] = \prod_{i \in [\ell]} [F_i, G_i],
\]
where $[F_i, G_i]$ is an interval of $\partitionPoset[n]$. Morover, given a $(\ell,n)$-partition forest $\b{G}\eqdef (G_1, \dots, G_\ell)$, any n-tuples of partition $(F_1, \dots, F_\ell)$ satisfying for any $1\leq i \leq \ell$ $F_i \leq_{\partitionPoset[n]} G_i $ is a $(\ell,n)$-partition forest as its intersection hypergraph is a subforest of the intersection hypergraph of $G$. 
Recall moreover that the M\"obius function is multiplicative.


Hence, we obtain that

\begin{align*}
\mobPol[\multiBraidArrangement] &=\sum_{G \in \forestPoset} y^{n-\ell n - 1 + \sum_{i=1}^\ell |G_i|} \prod_{i=1}^{\ell} \sum_{F_i \leq_{\partitionPoset[n]} G_i } \mu_{\partitionPoset[n]}(F_i,G_i) \, x^{n-\ell n - 1 + \sum_{i=1}^\ell |F_i|}, \\
 &=\sum_{G \in \forestPoset} y^{n-\ell n - 1 + \sum_{i=1}^\ell |G_i|} x^{(n-1)(1-\ell)} \prod_{i=1}^{\ell} \sum_{\pi_i \in \partitionPoset[|G_i|]}  \mu_{\partitionPoset[|G_i|]}(\pi_i,\hat{1}) \, x^{|\pi_i|-1}, 
\end{align*}
where $\hat{1}$ denotes the maximal element in $\partitionPoset[|G_i|]$ and $\pi_i$ is obtained from $F_i$ by merging elements in the same part of $G_i$.

Finally, we get:

\begin{equation}
\mobPol[\multiBraidArrangement]=\sum_{G \in \forestPoset} y^{n-\ell n - 1 + \sum_{i=1}^\ell |G_i|} x^{(n-1)(1-\ell)} \prod_{i=1}^{\ell} P_{|G_i|}(x)
\end{equation}

\begin{theorem}
The M\"obius polynomial of the $(\ell,n)$-braid arrangement~$\multiBraidArrangement$ is given by
\[
\mobPol[\multiBraidArrangement]=x^{(n-1)(1-\ell)} \sum_{G \in \forestPoset} y^{n-\ell n - 1 + \sum_{i=1}^\ell |G_i|}  \prod_{i=1}^{\ell} P_{|G_i|}(x).
\]
\end{theorem}

In terms of rainbow forests, this rewrites as

\begin{equation}
\mobPol[\multiBraidArrangement]=x^{(n-1)(1-\ell)} \sum_{G \in \rainbowForests} y^{n- 1 + |E(G)|}  \prod_{i=1}^{\ell} P_{n-|E(G,i)|}(x)
\end{equation}

To further simplify this expression, we now need to count the number of rainbow forests with a prescribed number of colored edges. We deal with this problem by extending the colored Pr\"{u}fer code of Theorem \ref{thm:verticesRefinedMultiBraidArrangement}


\begin{remark}
Bad news : I tried to first understand  how Pr\"{u}fer code works on (non colored) forests rooted in the minimal label of each connected component but no multiplicative formula is known (so the Prufer type proof does not seem to work either).
We get the sequence \OEIS{A138464}.
\end{remark}



\bibliographystyle{alpha}
\bibliography{diagonalsPermutahedra}
\label{sec:biblio}

\end{document}
                                               