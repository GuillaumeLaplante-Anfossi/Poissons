% !TEX root = ../Poissons.tex


\section{Vertices of the diagonal}
\label{s:vertices}

We are now interested in characterizing the pairs of vertices that occur in the diagonal, that is pairs of permutations $(\sigma_1,\sigma_2) \in \triangle$. 

\begin{thm} There exists $(I,J) \in D(n)$ such that $\forall k, |\sigma_1^1\cdots\sigma_1^k \cap I| \leq |\sigma_1^1\cdots\sigma_1^k \cap J|$ and $\forall l, |\sigma_1^1\cdots\sigma_1^l \cap I| \geq |\sigma_1^1\cdots\sigma_1^l \cap J|$ (diagonal condition) if and only if $\exists (I',J')=(\{i_1,\ldots,i_m\},\{j_1,\ldots,j_m\}) \in D(m)$, $m\leq n$, such that \[\sigma_1 \cap (I'\cup J')=j_1 i_1 j_2 i_2 \cdots j_n i_n \] and \[ \sigma_2 \cap (I'\cup J') = i_2 j_1 i_3 j_2 \cdots i_n j_{n-1} i_1 j_n \ , \] where $i_1 = \min (I' \cup J')$ (fish condition). 
\end{thm}

\begin{proof}
\begin{itemize}
\item If a pair of permutations $(\sigma_1, \sigma_2) \in   \mathfrak{S}_N^2$ satisfies the fish condition, then there exist two sets $I$ and $J$ of same cardinality such that $\min(I)<\min(J)$. Denoting $\sigma_1$ and $\sigma_2$ by two words of size $N$ $\sigma_1^1 \ldots \sigma_1^N$ and $\sigma_2^1 \ldots \sigma_2^N$, then the pair $((\sigma_1, \sigma_2), (I,J))$ satisfies that for any $k$ in $\llbracket 1;N\rrbracket$, $|\sigma_1^1 \ldots \sigma_1^k \cap J| \geq |\sigma_1^1 \ldots \sigma_1^k \cap I|$ and $|\sigma_2^1 \ldots \sigma_2^k \cap I| \geq |\sigma_2^1 \ldots \sigma_2^k \cap J|$, hence the diagonal condition.
\item We will now prove the converse. Let us presume that $(\sigma_1, \sigma_2)$ is a pair of permutations satisfying the diagonal condition for a pair of sets $(I,J) \in D(n)$, minimal for the inclusion of sets.
\begin{description}
\item[Case $n=1$] 
\end{description}
If $|I|=|J|=1$, then it follows directly from the diagonal condition above that ${\sigma_1}_{| I \cup J}=j_1 i_1$ and ${\sigma_1}_{|I \cup J}=i_1 j_1$, hence the fish condition is satisfied.
\begin{description}
\item[Case $n>1$] 
\end{description}
In this case, the proof is made by absurdum 
by considering the number of "well-placed" elements of $I$ and $J$ in $\sigma_1$ and $\sigma_2$. In what follows, for any set $E$, $\sigma^{E}_i$ will stands for $(\sigma_i)_{|E}$. We write also $n_{i,k}^E$ for the number of elements of $E$ in the $k$ first letters of $\sigma_i$. The main argument in each of the small proofs below is the same: if the permutations do not satisfy the pattern described above, then it is possible to find an appropriate pair of elements $(i,j)\in I \times J$ such that $(I-i,J-j)$ satisfies the diagonal condition, hence  contradicting the minimality of $(I,J)$.

We first prove that the leftmost element of $\sigma^{I}_1$ is $i_1$. Indeed, if it is not the case, we consider $i$, the leftmost element in $\sigma^{I}_1$ and $j$ the leftmost element in $\sigma^{J}_2$. The pair $(I-i,J-j)$ is in $D(n-1)$, as $i$ is different from $i_1$. Moreover, it is clear that the diagonal condition still holds for $((\sigma_1, \sigma_2), (I,J))$. As this would contradict the minimality of $(I,J)$, the leftmost element of $\sigma^{I}_1$ is $i_1$.

We then prove that $\sigma^{I \cup J}_1$ starts by $j_1 i_1$ and that this $j_1$ is exactly the leftmost element in  $\sigma^{J}_2$. On that purpose, we suppose that either  $i_1$ is preceeded by several elements of $J$ or that the unique element of $J$ is not the leftmost one in $\sigma^{J}_2$. We then adapt the previous argument by choosing $i$ to be the leftmost element in $\sigma^{I-\{i_1\}}_1$ and $j$ the leftmost element in $\sigma^{J}_2$. The pair $(I-i,J-j)$ is in $D(n-1)$. Let us briefly explain while  the diagonal condition would still be fulfilled in this case. If $j$ is after $i_1$ in $\sigma_1$, then the difference $n_{1,k}^{J-j}-n_{1,k}^{I-i}$ is greater than $n_{1,k}^{J}-n_{1,k}^{I}$ for any $k$, hence is non negative. If $j$ is before $i_1$ in $\sigma_1$, then by hypothesis, the difference $n_{1,k}^{J-j}-n_{1,k}^{I-i}$ is:
\begin{itemize}
\item strictly positive before $i_1$ an greater than $1$ just before $i_1$
\item non negative after $i_1$
\item increase between $i_1$ and $i$
\item is equal to $n_{1,k}^{J}-n_{1,k}^{I}$ after $i$,
\end{itemize} 
hence is always non negative.
Moreover, if $i$ is after $j$ in $\sigma_2$, the diagonal condition is clearly satisfied. If $i$ is before $j$, then the difference $n_{2,k}^{I-i}-n_{1,k}^{J-j}$ is:
\begin{itemize}
\item strictly positive before $j$ an greater than $1$ just before $j$
\item is equal to $n_{2,k}^{I}-n_{1,k}^{J}$ after $j$,
\end{itemize} 
hence is always non negative. In short, if $i_1$ is preceeded by several elements of $J$ or the unique element of $J$ is not the leftmost one in $\sigma^{J}_2$, we obtain a contradiction with the minimality of $(I,J)$.

Let us now consider the biggest $k\geq 1$ such that $\sigma^{I \cup J}_1$ begins with $j_1 i_1 j_2 i_2 \ldots j_k i_k$ and $\sigma^{I \cup J}_2$ begins with $i_2 j_1 i_3 j_2\ldots i_k j_{k-1} w j_k$, where $w$ is a word with letters in $I$. We want to show that $k=n$. Let us first remark that if $k=n$, $w=i_1$. If $1\leq k<n$, then the sets $\tilde{I}=I-\{i_1, \ldots, i_k\}$ and $\tilde{J}=J-\{j_1, \ldots, j_k\}$ are non empty. Let us choose $i_{k+1}$ to be the leftmost element in $\sigma^{\tilde{I}}_1$ and $j_{k+1}$ the leftmost element in $\sigma^{\tilde{J}}_2$. We thus have $\sigma^{I \cup J}_1=j_1 i_1 j_2 i_2 \ldots j_k i_k w' i_{k+1}\ldots$, where $w'$ is in $J$ and $\sigma^{I \cup J}_2= i_2 j_1 i_3 j_2\ldots i_k j_{k-1} w j_k w'' j_{k+1}\ldots $, where $w$ and $w'$ are words with letters in $I$. The pair $(I-i_{k+1},J-j_{k+1})$ is in $D(n-1)$. Following the study as in the previous case, $\sigma_1$ always satisfies the diagonal condition for $(I-i_{k+1},J-j_{k+1})$ and $\sigma_2$ satisfies it if and only if $w \neq i$. By minimality of $(I,J)$, we then have $w=i_{k+1}$. If $k+1=n$, we are done as the only possible word in $J$ is $j_{k+1}$, hence $w'=j_{k+1}$. Otherwise, we can choose $i_{k+2}$ to be the leftmost element in $\sigma_1^{\tilde{I}-i_{k+1}}$. Using the same reasoning as above, we show that $((\sigma_1, \sigma_2),(I-i_{k+2},J-j_{k+1}))$ satisfies the diagonal condition if and only if $w'\neq j_{k+1}$. To sum up, the only possibility for $(I,J)$ to be minimal is to have $k=n$, which implies the fish condition.
\end{itemize}
\end{proof}

\begin{corollary} For any pair of permutations $(\sigma_1, \sigma_2$, there exists $(I,J) \in D(n)$ such that $((\sigma_1, \sigma_2),(I,J))$ satisfies the diagonal condition if and only if there exists $(I',J') \in E(m)$, $m<n$ such that $((\sigma_1, \sigma_2),(I',J'))$ satisfies the fish condition, with 
\begin{multline}
E(m)=\{(I,J)\in D(m)| \min(J)<\min(I-\min(I)), |\llbracket 1; k \rrbracket \cap J| > |\llbracket 1; k \rrbracket \cap I| \\ \text{ if } |\llbracket 1; k \rrbracket \cap J| \geq 2 \text{ and } I \subsetneq \llbracket 1; k \rrbracket \}
\end{multline}
\end{corollary}

\begin{proof}
It follows directly from the fish condition: if the fish condition is satisfied, as inversions of $\sigma_1$ are included in inversions of $\sigma_2$, we get $j_{k-1},j_k<i_k$ for any $k>1$.
\end{proof}



