% !TEX root = ../Poissons.tex

\section*{Introduction} 
\label{s:introduction}

In this article, discrete geometry \emph{informs} higher algebra.

Some purely combinatorial results, whose study is motivated by higher algebra:
the combinatorially inclined reader can read the first part of this introduction, and go directly to Section 1.

Some higher algebraic results, obtained via discrete geometry: the algebraically inclined reader can read the second part of this introduction, and go directly to Section 2.

The reader who reads the entire article will start feeling the essence of what could be \emph{higher discrete geometry}, a new field of mathematics whose purpose is to use the theory of polytopes in order to study higher algebraic structures and homotopy theory.

\subsection*{Combinatorics}

\subsection*{Higher Algebra}

The purpose of this article is to study cellular diagonals on the permutahedra, which are cellular maps homotopic to the usual thin diagonal $\triangle : P \to P \times P, x \mapsto (x,x)$.
Such diagonals, and in particular coherent families that we call \emph{operadic} diagonals (see [DEF]), are of interest in algebraic geometry and topology: via the theory of Fulton--Sturmfels \cite{fultonIntersectionTheoryToric1997a}, they give explicit formulas for the cup product on Losev--Manin toric varieties \cite{losevNewModuliSpaces2000}; they define universal tensor products of homotopy operads, and in particular universal tensor products of homotopy associative permutads (shuffle algebras) \cite{LA21}; they allow for the definition of a coproduct on permutahedral sets, which are used to models of two-fold loop spaces \cite{SaneblidzeUmble04}; and their study is moreover needed to pursue the work of Baues aiming at defining explicit combinatorial models for higher iterated loop spaces \cite{bauesGeometryLoopSpaces1980}. 
Moreover, using the canonical projections to the multiplihedra and the associahedra, they define universal tensor product of $\Ainf$-algebras and $\Ainf$-morphisms \cite{MazuirLA22}.
\Guillaume{lien avec les matroides?}

The first cellular diagonal for the permutahedra was defined at the level of chains by S. Saneblidze and R. Umble in \cite{SaneblidzeUmble04}, we will call it the \emph{SU diagonal}. 
Cellular diagonals for the associahedra and the multiplihedra were also defined there, via projection. 
The first topological map for the permutahedra was given in \cite{LA21} -we will call it the \emph{LA diagonal}, where a general theory of cellular diagonals of polytopes was developed. 
The LA diagonal, however, is distinct from the SU diagonal at the cellular level \cite[Remark 3.19]{LA21}. 

computer program for the SU diagonal \cite{vejdemo-johanssonEnumeratingSaneblidzeUmbleDiagonal2007}; we give a much more effective computer program (however, without the signs)

corollary: le dernier article de SU!
corollary: IJ-description of SU diagonal
corollary: left and right shifts for LA, decomposition of the cube associated to LA diagonal



\subsection*{Conventions}
%We use the notations of \cite{LodayVallette12} for operads.

\Guillaume{Please make a new line for each sentence as to facilitate comparison between versions in GitHub}

\Guillaume{Would it be possible to avoid bold letters, use "slash emph\{\}", and also reduce to the maximum possible the use of indices; as soon as the context is clear, drop any extra index}



