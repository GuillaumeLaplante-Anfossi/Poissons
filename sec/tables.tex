% !TEX root = ../Poissons.tex

\section{Tables}
\label{s:tables}

In this section we present low dimensional computations of the enumeration results obtained above and we connect them to other known combinatorial objects. 

\begin{figure}[h]
\centerline{
\begin{tabular}{r|c}
\textbf{dim} & \textbf{0}  \\
\hline
\textbf{0} & 1  
\end{tabular}
\ \ \
\begin{tabular}{r|cc}
\textbf{dim} & \textbf{0} & \textbf{1}  \\
\hline
\textbf{0} & 3 & 1 \\
\textbf{1} & 1 &  
\end{tabular}
\ \ \
\begin{tabular}{r|ccc}
\textbf{dim} & \textbf{0} & \textbf{1} & \textbf{2}  \\
\hline
\textbf{0} & 17 & 12 & 1 \\
\textbf{1} & 12 &  6 & \\
\textbf{2} & 1 &  & 
\end{tabular}
\ \ \
\begin{tabular}{r|cccc}
\textbf{dim} & \textbf{0} & \textbf{1} & \textbf{2} & \textbf{3} \\
\hline
\textbf{0} & 149 & 162 & 38 & 1 \\
\textbf{1} & 162 & 150 & 24 & \\
\textbf{2} & 38 & 24 & & \\
\textbf{3} & 1 & & &
\end{tabular}
}
\caption{Number of pairs of faces in the cellular image of the diagonal $0$, $1$, $2$ and $3$-dimensional permutahedra.}
\label{t:dim1-3}
\end{figure}

\begin{figure}[h]
\centerline{
\begin{tabular}{r|ccccc}
\textbf{dim} & \textbf{0} & \textbf{1} & \textbf{2} & \textbf{3} & \textbf{4} \\
\hline
\textbf{0} & 1809 & 2660 & 1080 & 110 & 1 \\
\textbf{1} & 2660 & 3540 & 1200 & 80 & \\
\textbf{2} & 1080 & 1200 & 270 & & \\
\textbf{3} & 110 & 80 & && \\
\textbf{4} & 1 & & & &
\end{tabular}
\ \ \
\begin{tabular}{r|cccccc}
\textbf{dim} & \textbf{0} & \textbf{1} & \textbf{2} & \textbf{3} & \textbf{4} & \textbf{5} \\
\hline
\textbf{0} & 28399 & 52635 & 30820 & 6165 & 302 & 1 \\
\textbf{1} & 52635 & 90870 & 67580 & 7785 & 240 & \\
\textbf{2} & 30820 & 47580 & 20480 & 2160 & & \\
\textbf{3} & 6165 & 7785 & 2160 & && \\
\textbf{4} & 302 & 240 & & &&\\
\textbf{5} & 1 & & & &&
\end{tabular}
}
\caption{Number of pairs of faces in the cellular image of the diagonal $4$ and $5$-dimensional permutahedra.}
\label{t:dim4-5}
\end{figure}



\begin{figure}[h]
\centerline{\begin{tabular}{c|c|rrrrrrr|l}
\textbf{Pairs $(F,G) \in \Ima\triangle_{(P,\vec v)}$} & \textbf{Polytopes} & \textbf{0} & \textbf{1} & \textbf{2} & \textbf{3} & \textbf{4} & \textbf{5} & \textbf{6} & \textbf{\cite{OEIS}} \\
\hline
& \text{Associahedra} & 1 & 2 & 6 & 22 & 91 & 408 & 1938 & \OEIS{A000139}  \\
$\dim F + \dim G = \dim P$  & \text{Multiplihedra} & 1 & 2 & 8 & 42 & 254 & 1678 & 11790 &  to appear \\
  & \text{Permutahedra} & 1 & 2 & 8 & 50 & 432 & 4802 & 65536 &  \OEIS{A007334} \\
\hline
  & \text{Associahedra} & 1 & 3 & 13 & 68 & 399 & 2530 & 16965 &  \OEIS{A000260} \\
  $\dim F=\dim G =0$ & \text{Multiplihedra} & 1 & 3 & 17 & 122 & 992 & 8721 & 80920 & to appear \\
  & \text{Permutahedra} & 1 & 3 & 17 & 149 & 1809 & 28399 & 550297 &  \OEIS{A213507} 
\end{tabular}}
\caption{Number of pairs of faces in the cellular image of the diagonal of the associahedra, multiplihedra and permutahedra of dimension $0\leq \dim P \leq 6$, induced by any good orientation vector.}
\label{table:numerology}
\end{figure}



