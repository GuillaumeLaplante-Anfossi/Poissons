% !TEX root = ../Poissons.tex

\section{Combinatorics}

\subsection{Combinatorics of generically translated copies of the braid arrangement}
\label{sec:kBraidArrangement}

\subsubsection{Recollection on hyperplane arrangements}
\label{subsec:arrangements}

We first briefly recall classical results on the combinatorics of affine hyperplane arrangements, in particular the enumerative connection between their intersection posets and their face lattices due to T.~Zaslavsky~\cite{Zaslavsky}.

\begin{definition}
A (finite affine real) \defn{hyperplane arrangement} is a finite set~$\arrangement$ of affine hyperplanes in~$\R^d$.
\end{definition}

\begin{definition}
A \defn{region} of~$\arrangement$ is a connected component of~$\R^d \ssm \bigcup_{H \in \arrangement} H$.
A \defn{face} of~$\arrangement$ is the intersection of the closure of a region of~$\arrangement$ with an hyperplane of~$\arrangement$.
The \defn{face poset} of~$\arrangement$ is the poset~$\facePoset$ of faces of~$\arrangement$ ordered by inclusion.
The \defn{$f$-polynomial}~$\fPol$ and \defn{$b$-polynomial}~$\bPol$ of~$\arrangement$ are the polynomial
\[
\fPol \eqdef \sum_{k = 0}^d f_k(\arrangement) \, x^k
\qquad\text{and}\qquad
\bPol \eqdef \sum_{k = 0}^d b_k(\arrangement) \, x^k
\]
where~$f_k(\arrangement)$ denotes the number of $k$-dimensional faces of~$\arrangement$, while~$b_k(\arrangement)$ denotes the number of bounded $k$-dimensional faces of~$\arrangement$.
\end{definition}

\begin{definition}
A \defn{flat} of~$\arrangement$ is a non-empty affine subspace of~$\R^d$ that can be obtained as the intersection of some hyperplanes of~$\arrangement$.
The \defn{flat poset} of~$\arrangement$ is poset~$\flatPoset$ of flats of~$\arrangement$ ordered by reverse inclusion.
\end{definition}

\begin{definition}
%The \defn{characteristic polynomial}~$\chi_{\arrangement}(y)$ and the \defn{M\"obius polynomial}~$\mu_{\arrangement}(x,y)$ are the polynomials respectively defined by
%\[
%\charPol \eqdef \sum_{G \in \flatPoset} \mu_{\flatPoset}(\R^d, G) \, y^{\dim(G)}
%\quad\text{and}\quad
%\mobPol \eqdef \sum_{F \le G \in \flatPoset} \mu_{\flatPoset}(F,G) \, x^{\dim(F)} \, y^{\dim(G)}
%\]
The \defn{M\"obius polynomial}~$\mu_{\arrangement}(x,y)$ is the polynomials defined by
\[
\mobPol \eqdef \sum_{F \le G \in \flatPoset} \mu_{\flatPoset}(F,G) \, x^{\dim(F)} \, y^{\dim(G)}
\]
where~$\mu_{\flatPoset}(F,G)$ denotes the \defn{M\"obius function} on the flat poset~$\flatPoset$ defined as usual by
\[
\mu_{\flatPoset}(F, F) = 1
\qquad\text{and}\qquad
\sum_{F \le G \le H} \mu_{\flatPoset}(F,G) = 0
\]
for all~$F, G, H \in \flatPoset$.
\vincent{Whitney-number polynomial.}
\end{definition}

\begin{theorem}[{\cite[Thm.~A]{Zaslavsky}}]
The $f$-polynomial, the $b$-polynomial, and the M\"obius polynomial of an arrangement~$\arrangement$ are related by
\[
\fPol = (-1)^{\rank} \mobPol[-x][-1]
\qquad\text{and}\qquad
\bPol = (-1)^{\rank} \mobPol[-x][1].
\]
\vincent{Not sure of the powers of $-1$ here.}
\end{theorem}

\vincent{Whitney's theorem}

\begin{figure}
	\begin{overpic}[scale=1.5]{intersectionPoset}
		\put(453, -15){$1$}
		\put(310, 60){$-1$}
		\put(377, 60){$-1$}
		\put(444, 60){$-1$}
		\put(511, 60){$-1$}
		\put(578, 60){$-1$}
		\put(322, 195){$2$}
		\put(408, 195){$1$}
		\put(496, 195){$1$}
		\put(583, 195){$2$}
	\end{overpic}
	\caption{A hyperplane arrangement (left) and its intersection poset with its M\"obius function (right).}
\end{figure}

\subsubsection{The $(\ell,n)$-braid arrrangement}
\label{subsec:lnBraidArrangement}

\begin{definition}
For any integer~$\ell 1$, the \defn{braid arrangement}~$\braidArrangement$ is the arrangement of the hyperplanes~$\set{\b{x} \in \R^n}{x_i = x_j}$ for all~$1 \le i < j \le n$.
For any integer~$\ell 1$, the \defn{$(\ell,n)$-braid arrangement} is the arrangement obtained as the union of $\ell$ generically translated copies of the braid arrangement.
\end{definition}


\subsection{Enumerative results for any diagonal} 
\label{s:facets}

\Guillaume{Vincent, c'est \`a toi de jouer!}