% !TEX root = ../Poissons.tex

\newcommandx{\arrangement}[1][1 = A]{\mathcal{#1}} % arrangement

\section{Combinatorics of generically translated copies of the braid arrangement}
\label{sec:kBraidArrangement}

\subsection{Recollection on hyperplane arrangements}
\label{subsec:arrangements}

%We first briefly recall classical results on the combinatorics of affine hyperplane arrangements, in particular the enumerative connection between their intersection posets and their face lattices due to T.~Zaslavsky~\cite{Zaslavsky}.
%
%\begin{definition}
%A (finite affine) \defn{hyperplane arrangement} is a finite set~$\arrangement$ of affine hyperplanes in~$\R^d$.
%A \defn{flat} of~$\arrangement$ is a non-empty affine subspace of~$\R^d$ that can be obtained as the intersection of some hyperplanes of~$\arrangement$.
%The \defn{intersection poset} of~$\arrangement$ is poset~$\intersectionPoset$ of flats of~$\arrangement$ ordered by reverse inclusion.
%A \defn{region} of~$\arrangement$ is a connected component of~$\R^d \ssm \bigcup_{H \in \arrangement} H$.
%A \defn{face} of~$\arrangement$ is the intersection of the closure of a region of~$\arrangement$ with an hyperplane of~$\arrangement$.
%The \defn{face poset} of~$\arrangement$ is the poset~$\facePoset$ of faces of~$\arrangement$ ordered by inclusion.
%\end{definition}
%
%\begin{theorem}
%The number~$f_k(\arrangement)$ of $k$-dimensional faces of~$\arrangement$ is given by
%\[
%f_k(\arrangement) = \sum_{\substack{X \le Y \text{ in } \intersectionPoset \\ \dim(X) = k}} |\mu(X,Y)|.
%\]
%\end{theorem}

%In this section, we consider the arrangement

\section{Enumerative results for any diagonal} 
\label{s:facets}

\Guillaume{Vincent, c'est \`a toi de jouer!}